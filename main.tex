\documentclass[usenames,dvipsnames]{beamer}

\usepackage[portuguese]{babel}
\usepackage{xcolor}
\usepackage[utf8]{inputenc}
\usepackage{textpos}
\usepackage{tikz}
\usepackage{pgf}
\usepackage{eso-pic}
\usepackage{graphicx}

%Para fins de teste
\newcommand{\mylipsum}{just dummy text}
\newcommand{\myliplip}{\mylipsum{} \mylipsum{} \mylipsum{} \mylipsum}
\newcommand{\myblabla}{this is more serious or not}
\newcommand{\myblagf}{how i hear my girlfriend when i'm watching my footbal team? "bla bla bla bla"}
 
\usetheme{Boadilla}

%%%%%%%%%%%%%%%%%%%%%%%%%%%%%%%%%%%%%%%%%%%%%%%%%%%%%%%%%%%%%%%%%%%%%%%%%%
% Para inserir o logo no canto superior esquerdo, por definição está no
% canto inferior direito.
% Alteras os valores de pgfxy para alterar a posição do logo
%%%%%%%%%%%%%%%%%%%%%%%%%%%%%%%%%%%%%%%%%%%%%%%%%%%%%%%%%%%%%%%%%%%%%%%%%%%
\logo{

    %LOGO FCT
    \pgfputat{\pgfxy(-9.2,1.9)}{
        \pgfbox[right,base]{
            \includegraphics[width=2.5cm,keepaspectratio]{logos/logofct2}
        }
    }
    %LOGO UNL
    \pgfputat{\pgfxy(-0.1,2.4)}{
        \pgfbox[right,base]{
            \includegraphics[width=3.7cm,keepaspectratio]{logos/logounl}
        }
    }
}

%Comando para remover logos
\newcommand{\nologo}{\setbeamertemplate{logo}{}}

% Alteração das cores do footer para corresponder com as da fct
\setbeamercolor{author in head/foot}{fg=white, bg=LimeGreen!80!black}
\setbeamercolor{title in head/foot}{fg=white, bg=NavyBlue}
\setbeamercolor{date in head/foot}{fg=white, bg=Cyan!80!black}

%Alteração das cores para os titulos de cada slide
\setbeamercolor{frametitle}{fg=White, bg=NavyBlue}
%Alteração das cores para os subtitulos de cada slide
\setbeamercolor{framesubtitle}{fg=White}

%Alteração da formatação e cores dos ícones das listas e as suas formas
\setbeamertemplate{itemize item}{
    \color{Cyan!80!black}$\blacksquare$
}
\setbeamertemplate{itemize subitem} {
    \color{LimeGreen!80!black}$\blacktriangleright$
}

%Alteração da formatação e cores da listas enumeradas
\setbeamertemplate{enumerate items}[default]
\setbeamercolor{enumerate item}{fg=Cyan!80!black}
\setbeamercolor{enumerate subitem}{fg=LimeGreen!80!black}
  
%Alteração dos blocos de informação
\setbeamercolor{block title}{bg=NavyBlue!80!white,fg=white}
\setbeamercolor{block body}{bg=white!27!black,fg=white}

%Alterção das cores para os dados à aparecer no primeiro slide
\setbeamercolor{title}{fg=white}
\setbeamercolor{subtitle}{fg=white!80!black}
\setbeamercolor{author}{fg=white!80!black}
\setbeamercolor{date}{fg=white!80!black}

%Background canvas - se o valor do outer e inner for diferente queria um efeito visual
\usebackgroundtemplate{
    \begin{tikzpicture}
        \path [
            outer color = white!27!black, 
            inner color = white!27!black
        ]
        (0,0) rectangle (\paperwidth,\paperheight);
    \end{tikzpicture}
}

%Comando para remover o backgroud do canvas
\newcommand{\backgroundcolorwhite}{
    \usebackgroundtemplate{
        \begin{tikzpicture}
            \path [
                outer color = white, 
                inner color = white
            ]
            (0,0) rectangle (\paperwidth,\paperheight);
        \end{tikzpicture}
    }
}

%%%%%%%%%%%%%%%%%%%%%%%%%%%%%%%%%%%%%%%%%%%%%%%%%%%%%%%%%%%%%%%%%%%%%%%%%%
%
% Escrita do conteúdo dos slides
%
%%%%%%%%%%%%%%%%%%%%%%%%%%%%%%%%%%%%%%%%%%%%%%%%%%%%%%%%%%%%%%%%%%%%%%%%%%

%conteudo dentro dos [Qualquer Coisa] optional - Para aparecer no footer
\title[Relatório Estágio]
{
    \textbf{Titulo do trabalho/estudo/apresentação }
}
 
% Subtitulo
\subtitle{Subtitulo da apresentação. (ex: Preparação da Dissertação do Mestrado Integrado em Engenharia Informática)} 

%conteudo dentro dos [PrimeiroNome] optional - Para aparecer no footer
\author[John]{John Snow}

%Opção para mais que um autor e de diferentes instituições
% \author[Author, Anders] % (optional, for multiple authors)
% {F.~Author\inst{1} \and S.~Anders\inst{2}}
% \institute[Universities Here and There] % (optional)
% {
%   \inst{1}%
%   Institute of Computer Science\\
%   University Here
%   \and
%   \inst{2}%
%   Institute of Theoretical Philosophy\\
%   University There
% }

%conteudo dentro dos [] optional - Para aparecer no footer
\institute[FCT]
{
    \color{LimeGreen}F\color{black}aculdade de
    \color{NavyBlue!90!white}C\color{black}iências e 
    \color{Cyan}T\color{black}ecnologia 
    \\ 
    \color{white!60!black}Universidade \color{green!65!black}Nova \color{white!60!black}de Lisboa
}

%conteudo dentro dos [] optional - Para aparecer no footer
\date[\today]{\today}


%Inicio do documento
\begin{document}

%Criação do primeiro slide
\maketitle
%Remoção dos logos para as restantes slides
\nologo
\backgroundcolorwhite
%Inclusão dos ficheiros que compões os restantes slides
\begin{frame}
    \frametitle{Introdução}

    \begin{itemize}[<+->]
        \item Motivação
        \item Problemas
        \item Objetivos
    \end{itemize}

    \only<1>{
     \begin{block}{Motivação}
        \begin{enumerate}
            \item \myliplip
            \begin{enumerate}
                \item \mylipsum
                \item \mylipsum
            \end{enumerate}
        \end{enumerate}
     \end{block}
    }
    
    \only<2>{
     \begin{block}{Problemas}
        \begin{enumerate}
            \item \myliplip
        \end{enumerate}
     \end{block}
    }
    
    \only<3>{
     \begin{block}{Objetivos}
        \begin{enumerate}
            \item \myliplip
        \end{enumerate}
     \end{block}
    }
    
    

\end{frame}
 
    %Estes ficheiro pertencem à introdução por isso a identação (Não obrigatório)
    %Primeiro slide da Motivação
\begin{frame}
\frametitle{Introdução}
    \framesubtitle{Motivação}
        
        \begin{itemize}[<+->]
            \item \myliplip
            \item \myblabla
            \item \myliplip
        \end{itemize}
\end{frame}

%Primeiro slide da Motivação
\begin{frame}
    \frametitle{Introdução}
        \framesubtitle{Motivação}
            \begin{figure}
                \centering
                \begin{minipage}{.5\textwidth}
                    \centering
                    \includegraphics[width=0.85\linewidth]{figures/science.png}
                \end{minipage}%
                \begin{minipage}{.5\textwidth}
                    \centering
                    \includegraphics[width=0.85\linewidth]{figures/science.png}
                \end{minipage}
            \end{figure}
\end{frame}
    \begin{frame}
\frametitle{Introdução}
    \framesubtitle{Problema}
        
    \begin{itemize}[<+->]
            \item \myliplip
            \item \myliplip
            \begin{itemize}
                \item \mylipsum
            \end{itemize}
    \end{itemize}
\end{frame}
 
    \begin{frame}
\frametitle{Introdução}
    \framesubtitle{Objetivo}
        
        Also can write something here \myblabla \mylipsum
        \begin{itemize}[<+->]
            \item \myliplip
            \item \myliplip
            \item \myblabla
            \item \mylipsum
        \end{itemize}
\end{frame}
 
\begin{frame}
\frametitle{Estado de Arte}
BLA BLA BLA BLA BLA
BLA BLA BLA BLA BLA
BLA BLA BLA BLA BLA
BLA BLA BLA BLA BLA
BLA BLA BLA BLA BLA
\end{frame}
 
\begin{frame}
\frametitle{Trabalho Proposto}
BLA BLA BLA BLA BLA
BLA BLA BLA BLA BLA
BLA BLA BLA BLA BLA
BLA BLA BLA BLA BLA
BLA BLA BLA BLA BLA
\end{frame}
 

\end{document}
