\documentclass[usenames,dvipsnames]{main}

\usepackage[portuguese]{babel}
\usepackage{xcolor}
\usepackage[utf8]{inputenc}
\usepackage{textpos}
\usepackage{tikz}
\usepackage{pgf}
\usepackage{eso-pic}
\usepackage{graphicx}

\usetheme{Boadilla}

%%%%%%%%%%%%%%%%%%%%%%%%%%%%%%%%%%%%%%%%%%%%%%%%%%%%%%%%%%%%%%%%%%%%%
%                                                                   %
% -> The logo command is a predefined command of beamer class       %
%                                                                   %
% -> It's possible to insert a lot of logos. To do it we just       %
%  needed to insert the command \newlogo inside the predefined      %
%  \logo command and add new logos to the first slide               %
%                                                                   %
% -> Explanation of \newlogo{1}{2}{3}{4}:                           %
%                                                                   %
%    1) The path to logo (ex: logos/logofct2)                       %
%                                                                   %
%    2) Size (numeric) of logo, they will maintan aspect ratio      %
%       (ex:2.5)                                                    %
%                                                                   %
%    3) Horizontal position (numeric) in slide (ex: -9.2)           %
%                                                                   %
%    4) Vertical position (numeric) in slide (ex: 1.9)              %
%                                                                   %
%%%%%%%%%%%%%%%%%%%%%%%%%%%%%%%%%%%%%%%%%%%%%%%%%%%%%%%%%%%%%%%%%%%%%
\logo{
    %Insert FCT logo
    \newlogo{logos/logofct2}{2.2}{-55}{80}
    %Insert UNL logo
    \newlogo{logos/logounl}{3.2}{-85}{30}
    %Insert head icon
    \newlogo{figures/headicon}{3.5}{-342}{28}
}


%%%%%%%%%%%%%%%%%%%%%%%%%%%%%%%
%                             %
% Stylised Content -          %
%                             %
%%%%%%%%%%%%%%%%%%%%%%%%%%%%%%%
\fctfooter
\bluewhitetitle
\firstpage
\changelisticons


%%%%%%%%%%%%%%%%%%%%%%
%                    %
% Begin presentation %
%                    %
%%%%%%%%%%%%%%%%%%%%%%

% The content between [] in all fields is optional (to show on footer)

% Principle title of presentation
\title[Relatório Estágio]
{
    \textbf{Titulo do trabalho/estudo/apresentação }
}
 
% Subtitule of presentation
\subtitle{Subtitulo da apresentação. (ex: Preparação da Dissertação do Mestrado Integrado em Engenharia Informática)} 


%%%%%%%%%%%%%%%%%%%%%%%%%%%%%%%%%%%%%
%                                   %
% Author name (case for one author) %
%                                   %
%%%%%%%%%%%%%%%%%%%%%%%%%%%%%%%%%%%%%
\author[John Snow]{John Snow}

%%%%%%%%%%%%%%%%%%%%%%%%%%%%%%%%%%%%%%%%%%%%%%%%%%%%%%%%%%%%%
%                                                           %
% If the presentation have two or more author               %
%                                                           %
%%%%%%%%%%%%%%%%%%%%%%%%%%%%%%%%%%%%%%%%%%%%%%%%%%%%%%%%%%%%%
% \author[Author, Anders] % (optional, for multiple authors)
% {F.~Author\inst{1} \and S.~Anders\inst{2}}
% \institute[Universities Here and There] % (optional)
% {
%   \inst{1}%
%   Institute of Computer Science\\
%   University Here
%   \and
%   \inst{2}%
%   Institute of Theoretical Philosophy\\
%   University There                                             
% \}                                                            
%%%%%%%%%%%%%%%%%%%%%%%%%%%%%%%%%%%%%%%%%%%%%%%%%%%%%%%%%%%%%%

% The name of institute that has realize the study
\institute[FCT]
{
    \color{LimeGreen}F\color{black}aculdade de
    \color{NavyBlue!90!white}C\color{black}iências e 
    \color{Cyan}T\color{black}ecnologia 
    \\ 
    \color{white!60!black}Universidade \color{green!65!black}Nova \color{white!60!black}de Lisboa
}

% The date (\today command on ever compile code will put the current date)
\date[\today]{\today}

%%%%%%%%%%%%%%%%%%%%%%%
%  Begin document  %
%%%%%%%%%%%%%%%%%%%%%%%
\begin{document}

% Create the first slide with the previous content describe 
\maketitle

% Remove the logos for the next slides 
% (it's possible to ``create'' a new logo with command \newlogo in different parts of the canvas)
\nologo
%\removebackgroundcanvas

% Include files por presentatio
\begin{frame}
    \frametitle{Introdução}

    \begin{itemize}[<+->]
        \item Danger
        \item Info
        \item Success
        \item Code
    \end{itemize}

    \blockdanger
    \only<1>{
     \begin{block}{Danger}
        \begin{enumerate}
            \item this is a block danger 1
            \begin{enumerate}
                \item this is a block danger 1.1
            \end{enumerate}
        \end{enumerate}
     \end{block}
    }
    
    \blockinfo
    \only<2>{
     \begin{block}{Info}
        \begin{enumerate}
            \item This is a block info
        \end{enumerate}
     \end{block}
    }
    
    \blocksuccess
    \only<3>{
     \begin{block}{Success}
        \begin{enumerate}
            \item This is a block success
        \end{enumerate}
     \end{block}
    }
    
    \blockcode
    \only<4>{
     \begin{block}{Code}
        \begin{enumerate}
            \item This is a block code
            \begin{enumerate}
                \item The second enumerate is different that the befor
            \end{enumerate}
        \end{enumerate}
     \end{block}
    }

\end{frame}
 
    % This are idented for organization, no required (implies that the slides above are related with introduction )
    \include{slides/2_motivation}
    \include{slides/3_problem}
    \include{slides/4_objective}
\begin{frame}
\frametitle{Estado de Arte}
	\myblabla \\
	\myblagf
\end{frame}
 
\include{slides/6_proposedwork}

\end{document}

%%%%%%%%%%%%%%%%%%%%%%%%%%%%%%%%%%%%%%%%%%%%%%%%%%%%%%%%%%%%%%%%%%%%%%%%%%
%                                                                        
% To the Textpad (MacOS) recognize and import the files in folder slides 
%                                                                         
%%%%%%%%%%%%%%%%%%%%%%%%%%%%%%%%%%%%%%%%%%%%%%%%%%%%%%%%%%%%%%%%%%%%%%%%%%
\include{slides}